\documentclass{article}

\usepackage{amsmath,amssymb,amsthm,bm}
\usepackage{graphicx}
\usepackage{subcaption}
\usepackage{float}
\usepackage{xifthen}
\usepackage{hyperref}
\hypersetup{
    colorlinks=true,
    linkcolor=blue,
    %filecolor=magenta,
    urlcolor=cyan
}

\newcommand{\der}[3][1]{
	\frac{{\text{d}^{
		\ifthenelse{\equal{#1}{1}}{}{\,#1}
	}
	{#2} }}
	{\text{d} {#3}^{
		\ifthenelse{\equal{#1}{1}}{}{#1}
	}
	}
}

\newcommand{\pder}[3][1]{
	\frac{{\partial^{
		\ifthenelse{\equal{#1}{1}}{}{\,#1}
	}
	{#2}}}
	{\partial {#3}^{
		\ifthenelse{\equal{#1}{1}}{}{#1}
	}
	}
}

\newcommand{\R}{\mathbb{R}}

\renewcommand{\vec}[1]{\bm{#1}}

\newcommand{\norm}[2][2]{\left|\left| #2 \right|\right|_{#1}}


\title{Numerical Simulation of Wave Scattering Off Antenna}
\author{Jerome \\
University of Delaware, Department of Mathematical Sciences}
\date{}

\begin{document}

\maketitle

\tableofcontents

\newpage

\section{Introduction}

The goal in this project is to solve the following 2 dimensional equation:

\begin{equation}
  \label{eq:wave-eq-unscaled}
  \begin{split}
  c^2 \nabla^2 \psi & = \pder[2]{\psi}{\tilde{t}} \\
  x \in \Omega & = \left[-X,X\right] \times \left[-Y,Y\right],
  t \in \left[0,T\right] \\
  \left.\psi\right|_{\partial \Omega} = 0, & \quad
  \psi(\vec{x},0) = f(\vec{x}), \quad \pder{\psi}{t}(\vec{x},0) = g(\vec{x})
  \end{split}
\end{equation}

Where $c > 0$ is the speed of propagation of the wave.
We will solve this equation in the presence of a reflecting antenna.
The antenna will be an arc, in $\R^2$ represented by $A(x,y) = 0$.
Since the antenna will be reflecting, this implements a further condition on
$\psi$:

$$\psi(x,y,t) = 0 \quad \text{on} \quad A(x,y) = 0$$

This reflection is what causes the scattering and so generates the
beam pattern of the antenna.
Our goal is to calculate this beam pattern given the antenna shape.

For normalization purposes, we can let $t = \frac{1}{c}\tilde{t}$, which
reduces Eq. \ref{eq:wave-eq-unscaled} to

\begin{equation}
  \label{eq:wave-eq}
  \nabla^2 \psi = \pder[2]{\psi}{t}
\end{equation}

Note the boundaries on $t$ in Eq. \ref{eq:wave-eq-unscaled} have no "$\sim$".
This is because once $t$ is normalized, we may define new parameters which
take into account this normalization.

In an analytic solution, we would allow the boundaries of the problem to be
at infinity, as this is the physical case.  In a simulation however this is
not possible.  To deal with this we will choose $x \in \left[-X,X\right]$ and
$y \in \left[-Y,Y\right]$ such that the wave front will not reach the
boundary within the allotted solution time ($t \in \left[0,T\right]$)

\section{Methods}

\subsection{Discretization}

We discretize the system in accordance with Method of Lines (MoL).  We choose
$M_x$ points in the $x$ direction and $M_y$ points in the $y$ direction.
This gives a spacial discretization of

$$x_i = -X + \frac{2X}{M_x}i, \quad i = 0, 1, ..., M_x, \quad
y_j = -Y + \frac{2Y}{M_y}j, \quad j = 0, 1, ..., M_y$$

For simplicity we will denote $h_x = \frac{2X}{M_x}$ and similarly for $h_y$.
These are the spacings in the $x$ and $y$ directions respectively.
With the spacial discretization, we transform $\psi(x,y,t)$ into a matrix
which is a function only of $t$ such that

$$\Psi_{ij}(t) = \psi(x_i,y_j,t)$$

To approximate the spacial derivatives, we will use a second order formula.
For any point on the interior, $1 \leq i \leq M_x - 1$
and $1 \leq j \leq M_y - 1$, all neighbors exist.
This enables the approximation

$$\left.\pder[2]{\psi}{x}\right|_{x_i} =
\frac{1}{h_x^2} \left(\Psi_{i-1,j} - 2 \Psi_{i,j} + \Psi_{i+1,j}\right)
+ O(h_x^2), \quad
\left.\pder[2]{\psi}{y}\right|_{y_j} = \frac{1}{h_y^2}
\left(\Psi_{i,j-1} - 2 \Psi_{i,j} + \Psi_{i,j+1}\right) + O(h_y^2)$$

For the end points, let us consider $\pder[2]{\psi}{x}$ at $x_0$.
Then we can construct the following:

$$\left.\pder[2]{\psi}{x}\right|_{x_0} =
\frac{1}{h_x^2}\left(
\Psi_{i,j} - \frac{5}{2} \Psi_{i+1,j} +
2 \Psi_{i+2,j} - \frac{1}{2} \Psi_{i+3,j}\right) + O(h_x^2)$$

A similar formula may be constructed for $\pder[2]{\psi}{x}$ at $x_{M_x}$
as well as for the $y$ counterparts.  Putting this information together
we can construct an operator for $\Psi$ which will construct the spacial partial
derivatives

\begin{equation}
  \label{eq:differentiation-matrix}
  \pder[2]{\psi}{x} \approx D_xx \Psi, \quad D_xx = \frac{1}{h_x^2}
  \begin{bmatrix}
  1 & -\frac{5}{2} & 2 & -\frac{1}{2} \\
  1 & -2 & 1 \\
  & 1 & -2 & 1 \\
  & & \ddots & \ddots & \ddots \\
  & & & 1 & -2 & 1 \\
  & & \frac{1}{2} & -2 & \frac{5}{2} & -1 \end{bmatrix}
\end{equation}

With a similar construction for $D_{yy}$.  It should be noted however that
while $D_{xx}$ needs to operate on the rows of $\Psi$ (as this is where $x$
changes), $D_{yy}$ must therefore operate on the columns.  Therefore we have

$$\pder[2]{\psi}{y} \approx D_{yy} \Psi^T$$

However since we will add the two together for the Laplacian, we want $\Psi$ to
have the same orientation in each equation.  Therefore taking the
transpose of the above and summing gives an equation for the
Laplacian of $\psi$:

\begin{equation}
  \label{eq:discrete-laplacian}
  \nabla^2 \psi \approx D_{xx} \Psi + \Psi D_{yy}^T
\end{equation}

Finally putting this together with Eq. \ref{eq:wave-eq} gives:

\begin{equation}
  \label{eq:method-lines-ode}
  \der[2]{}{t} \Psi = D_{xx} \Psi(t) + \Psi(t) D_{yy}^T
\end{equation}

This now gives a second order ODE system which can be integrated.

\subsection{ODE Integration}

To integrate Eq. \ref{eq:method-lines-ode} we will use the
St\"ormer Verlet method.

The Verlet method is defined as follows:
suppose $\der[2]{}{t} u = f(t,u)$, with $t \in \left[0,T\right]$ and the
following are initial conditions:
$u(0) = u_0, u'(0) = v_0$.  We use $N$ time nodes, and define
$\tau = \frac{T}{N}$.  Let $u_k \approx u(t_k)$ where $t_k = k\tau$,
$k = 0, 1, ..., N$. Then the Verlet method proceeds as follows:

\begin{itemize}
  \item Set $u_1 = u_0 + \tau v_0 + \frac{\tau^2}{2} f(0,u_0)$
  \item iterate by
  $$u_{k+1} = 2u_k - u_{k-1} + \tau^2 f(t_k,u_k)$$
\end{itemize}

It should be noted that the Verlet method is second order accurate
\textbf{citation needed}.  We can apply this to our problem given that
$\psi(x,y,0) = f(x,y)$ and $\pder{\psi}{t}(x,y,0) = g(x,y)$.  We start
by creating matrix forms of $f$ and $g$ respectively:

$$F_{ij} = f(x_i,y_j), \quad G_{ij} = g(x_i,y_j)$$

Then we apply the Verlet Method.  Let $\Psi^{(k)} \approx \Psi(t_k)$:

\begin{equation}
    \label{eq:verlet-integration}
    \begin{split}
      \Psi^{(0)} & = F \\
      \Psi^{(1)} & = F + \tau G +
      \frac{\tau^2}{2} \left(D_{xx} F + F D_{yy}^T\right) \\
      \Psi^{(k+1)} & = 2\Psi^{(k)} - \Psi^{(k-1)} +
      \tau^2 \left(D_{xx} \Psi^{(k)} + \Psi^{(k)}D_{yy}^T \right)
    \end{split}
\end{equation}

Where again $\tau = \frac{T}{N}$.  At each Verlet step, we will set
$\psi = 0$ along $A(x,y) = 0$.  This will induce a reflection from the Antenna
and therefore allow us to calculate the beam pattern.  It should be noted that
due to the discretization in $x$ and $y$, we may not have exactly $A(x,y) = 0$
where intended.  To deal with this, we allow a tolerance on the value $A$.
If $|A| < \epsilon$, then we say $A \approx 0$ and force a reflection.


\section{Results}

\section{Conclusions}

\end{document}
